\subsubsection{Major Comments}

\begin{revquote}
1. Matching claims to results
The paper states that VALINOR is as accurate or more accurate than the RI-EKF, and much faster. This is true for some parts, like tilt and lateral position. But for other parts, like yaw and lateral velocity, the results are about the same or slightly worse. These points are mentioned, but only briefly. It would be better to clearly say which parts are improved and which are not. This would make the claims more honest and easier to trust.
\end{revquote}

We thank the reviewer for this comment and agree that our explanations required greater detail to avoid ambiguity. We have revised the text to clarify where our method outperforms and underperforms compared to the RI-EKF, and to more clearly state the criteria an estimator must meet for our intended use case (remote control / local task planning and execution). This revision highlights that each approach has its own relative strengths with regard to these criteria. As it is difficult to judge which strengths are more valuable, we conclude that both estimators are well suited to our use case, as each demonstrates strong estimation performance. Since the improvements brought by {\scshape Valinor} were significant and consistent, we reiterated them in the conclusion without claiming overall superiority in estimation accuracy.

\begin{revquote} \hypertarget{Comment 2 Rev 3}{}
2. Experiment scope and discussion of general cases
The experiments were done on two humanoid robots with tasks like walking on flat ground and multi-contact motion. This shows that the method works in practice. Still, it would help to briefly discuss how the method might behave in more difficult cases, like when foot contact is missed or the surface is slippery. This isn't critical to the paper, but adding a few sentences about assumptions and possible failure cases would make the paper feel more complete.
\end{revquote}

We thank the reviewer for this comment, and we should indeed have elaborated on such problems. Our method would indeed be affected by contact slippage since our position and yaw estimation relies on Leg odometry. The latter considers the contact pose fixed in the world, thus contact slippage would bot be ``observed'' and our position and yaw estimates would then drift from the actual ones. It is a well-known limitation of Leg odometry, which methods like the RI-EKF (\cite{Hartley2020RIEKF}) try to mitigate by using Leg odometry along Inertial odometry and appending the contact position to the state such that it is slightly corrected over time. In any case, such drift is impossible to recover without exteroceptive measurements. A good example is Figure~\ref{fig:traj_friends}, which shows that during Experiment 1, both {\scshape Valinor} and the RI-EKF drifted due to slippage of contacts, which at times during the experiment was significant enough to be visible to the naked eye.
Our tilt estimate, provided by the Tilt Observer, is expected to be less impacted by slippage. Slippage would indeed cause high-frequency disturbances in the velocity measurement $\boldsymbol{y}_v$, but these are filtered within the Tilt Observer's complementary filter. You can notably see that the error on our tilt estimate and its standard deviation are low during Experiment 1, despite the slippage.

However, in the use case we are targeting with {\scshape Valinor}, where the robot is controlled remotely or when tasks are planned in the robot's local task space, slippage is less problematic since a drift in the absolute pose does not affect the tasks. Building on your comment, we have now clarified at the end of our abstract which controllers our estimator is intended for. We have also reworked the whole section~\ref{sec:exps} (Experimental Evaluation) to insist on the fact we address this use case, and we state clearly the criteria an estimator should meet to be appropriate in this context. We now also discuss this limitation in the conclusion of the paper, noting that for task that need the global pose to be estimated accurately, methods leveraging exteroceptive sensors would, of course be more suitable. 

Concerning contact failure, we can imagine two scenarios:
\begin{enumerate}
    \item a contact is mistakenly not detected. In that case, it would simply not impact the estimation, and the remaining contacts would be sufficient to perform the estimation (our method works from one contact).
    \item no contact is detected. In that case, the position and yaw could no longer be updated and we would need to consider it constant. Concerning the tilt estimation, we would no longer be able to compute the velocity measurement $\boldsymbol{y}_v$ required by the Tilt Observer, since it relies on the anchor point kinematics, which itself originates from those of the contacts. The correction of the tilt estimate by this velocity measurement would be lost, thus the tilt's convergence properties too, but it would still be obtained by the integration of the gyrometer's measurement. It would thus drift slightly due to noise integration, but such a state would have to be maintained during many iterations in order for this drift to become critical. However, as soon as a contact is set again, the Tilt Observer would recover and the local linear velocity and tilt would converge again, as proven in~\cite{benallegue2020LyapunovStableOrientationEstimatorHumanoids}.
    This would probably be the worst scenario for {\scshape Valinor}, but it would actually be also problematic for any other Leg Inertial odometry, including the RI-EKF, since contacts are essential to provide the observability of the tilt and the linear velocity~\cite{bloesch2013FusionLegKineAndImu}. Similarly to the Tilt Observer, these variables would drift due to the integration of noise in the IMU sensors. Since the IMU is involved in their full pose estimation, their yaw estimate would drift too, and more critically, their position estimate would drift quadratically in time due to the double integration, unless they prefer considering it constant.
\end{enumerate}

The last difficult scenario we imagined is the case of biased IMU measurements. For this scenario, we kindly refer you to the reponse we did to comment number 4 by Reviewer 1, who was curious at how our method would behave in that case.

To conclude this reponse, if you are interested in the challenges associated with proprioceptive odometry, we recommend reading our related work, "The Kinetics Observer: A Tightly Coupled Estimator for Legged Robots”, Demont et al. 2024. In this paper, we further develop this topic, provide a broader overview of the state of the art, and propose a solution to mitigate even more contact slippage, notably by using the force / torque measurements and the robot's dynamic model. In Table 3 (p. 13) / Section 5.4 of that paper, we also present the estimation results of the RI-EKF in case of slippage.



\subsubsection{Minor Comments}
\begin{revquote}
1. Please check for small typos like "instantanous" (should be "instantaneous") or "accross" (should be "across").
2. Some phrases are unclear. For example, "we decided to leave them changed" is probably meant to say "we decided not to change them."
\end{revquote}

We thank the reviewer for correcting us, and apologize for these typos, which we have now fixed.

\begin{revquote}
3. Figures like Fig. 4 and 5 should include clear axis labels and units (e.g., seconds, degrees).
\end{revquote}

We thank the reviewer for this comment and agree that we should have used the commonly used notations in the figures. Following your comment, we are now using [seconds] for the Time label. Since labels would overlap in Figure~\ref{fig:pose_rhps1} if we used the [degrees] and [meters] notations, we have adopted the [deg] and [m] notations, which has been used in well-known works on odometry for legged robots~\cite{wisth2022vilens, Lin2005ALegConfigurationMeasSystemHexapod}. 